%%
%%      coprime.tex
%%      author: Sungbae Jeong
%%      coprime library for LuaPlainTex
%%
%%      version 0.2.0
%%

\def\ifundefined#1{\expandafter\ifx\csname#1\endcsname\relax}
\ifundefined{luatexversion}
    \errmessage{Use luatex to compile this}
\fi

\input ams-math
\input kotexplain
\input tikz

\catcode`@=11

\font\tenamsb=msbm10 \font\sevenamsb=msbm7 \font\fiveamsb=msbm5
\newfam\bbfam
\textfont\bbfam=\tenamsb
\scriptfont\bbfam=\sevenamsb
\scriptscriptfont\bbfam=\fiveamsb

\font\teneufm=eusm10 \font\seveneufm=eusm7 \font\fiveeufm=eusm5
\newfam\eufmfam
\textfont\eufmfam=\teneufm
\scriptfont\eufmfam=\seveneufm
\scriptscriptfont\eufmfam=\fiveeufm
\def\eufm{\fam\eufmfam}

\let\scr=\script

\directlua{
function __DefineFont__(fontName, fontSize, latinName, koreanName)
    local latinFontName = [[\string\]] .. fontName .. "latin"
    local hangulFontName = [[\string\]] .. fontName .. "hangul"
    tex.print([[\string\font]] .. latinFontName .. "=" .. latinName .. " at " .. fontSize .. "pt")
    tex.print([[\string\sethangulfont]] .. hangulFontName .. "=" .. koreanName .. " at " .. fontSize .. "pt")
    tex.print([[\string\def\string\]] .. fontName .. "{" .. latinFontName .. hangulFontName .. "}")
end
}
\def\definefont#1#2#3#4{\directlua{__DefineFont__("#1",#2,"#3","#4")}}

\definefont{titlefnt}{16}{cmbx10}{UnBatangBold}
\definefont{authorfnt}{11}{cmr10}{UnBatang}
\definefont{datefnt}{11}{cmr10}{UnBatang}
\definefont{sectionfnt}{14}{cmbx10}{UnBatangBold}
\definefont{subsectionfnt}{11.5}{cmbx10}{UnBatangBold}

\definefont{tiny}{5}{cmr10}{UnBatang}
\definefont{footnotesize}{8}{cmr10}{UnBatang}
\definefont{small}{9}{cmr10}{UnBatang}
\definefont{normalsize}{10}{cmr10}{UnBatang}
\definefont{large}{12}{cmr10}{UnBatang}
\definefont{Large}{14.4}{cmr10}{UnBatang}
\definefont{LARGE}{17.28}{cmr10}{UnBatang}
\definefont{huge}{20.74}{cmr10}{UnBatang}
\definefont{Huge}{24.88}{cmr10}{UnBatang}

\def\pdfcolorstack{\pdfextension colorstack}
\def\pdfcolorstackinit{\pdffeedback colorstackinit}

\def\today{\ifcase\month\or
  January\or February\or March\or April\or May\or June\or
  July\or August\or September\or October\or November\or December\fi
  \space\number\day, \number\year}
\def\title#1#2#3{\directlua{
    tex.print([[\string\centerline{\string\titlefnt\space #1}]])
    tex.print([[\string\kern 0.8em]])
    tex.print([[\string\centerline{\string\datefnt\space #2}]])
    tex.print([[\string\kern 0.8em]])

    if "#3" == "" then
    tex.print([[\string\centerline{\string\datefnt\space\string\today}]])
    else
    tex.print([[\string\centerline{\string\datefnt\space #3}]])
    end

    tex.print([[\string\kern 3em]])
}}

\edef\pdfhorigin{\pdfvariable horigin}
\edef\pdfvorigin{\pdfvariable vorigin}

\newcount\captioncnt \newcount\eqnocnt \newcount\sectioncnt \newcount\subsectioncnt
\newif\ifsecti@nused

\def\section#1{
    \vskip 1pc
    \subsectioncnt=0
    \advance\sectioncnt by 1
    \noindent{\sectionfnt\the\sectioncnt.\space#1}\par
    \vskip 1em
    \secti@nusedtrue
}

\def\subsection#1{
    \ifsecti@nused\else
    \vskip 1em
    \fi
    \secti@nusedfalse
    \captioncnt=0
    \eqnocnt=0
    \advance\subsectioncnt by 1
    \noindent{\subsectionfnt\the\sectioncnt.\the\subsectioncnt.\space#1}\par
    \vskip 1ex
}

% simple implementation for resizing elements
\protected\def\pdfsetmatrix{\pdfextension setmatrix}
\protected\def\pdfsave{\pdfextension save\relax}
\protected\def\pdfrestore{\pdfextension restore\relax}

% NOTE: #1 must be an integer
\def\resizebox#1#2{\setbox0=\hbox{#2}%
\setbox1=\hbox{\pdfsave\pdfsetmatrix{#1 0 0 #1}\rlap{#2}\pdfrestore}%
\ht1=#1\ht0 \wd1=#1\wd0 \box1\hbox to0pt{}}

\mathchardef\colorcnt=\pdfcolorstackinit page {0 g 0 G}
\def\colorpop{\pdfcolorstack\colorcnt pop}
\def\colorpush#1{\pdfcolorstack\colorcnt push {#1 k #1 K}}
\def\colorset#1{\pdfcolorstack\colorcnt set {#1 k #1 K}}
\def\grayscalepush#1{\pdfcolorstack\colorcnt push {#1 g #1 G}}
\def\grayscaleset#1{\pdfcolorstack\colorcnt set {#1 g #1 G}}

\def\Red{\colorpush{0 1 1 0}\aftergroup\colorpop}
\def\Blue{\colorpush{1 1 0 0}\aftergroup\colorpop}
\def\Green{\colorpush{1 0 1 0}\aftergroup\colorpop}
\def\Cyan{\colorpush{1 0 0 0}\aftergroup\colorpop}
\def\Magenta{\colorpush{0 1 0 0}\aftergroup\colorpop}
\def\Yellow{\colorpush{0 0 1 0}\aftergroup\colorpop}
\def\Black{\grayscalepush{0}\aftergroup\colorpop}
\def\White{\grayscalepush{1}\aftergroup\colorpop}
\def\RedCustom#1{\colorpush{0 #1 #1 0}\aftergroup\colorpop}
\def\BlueCustom#1{\colorpush{#1 #1 0 0}\aftergroup\colorpop}
\def\GreenCustom#1{\colorpush{#1 0 #1 0}\aftergroup\colorpop}
\def\CyanCustom#1{\colorpush{#1 0 0 0}\aftergroup\colorpop}
\def\MagentaCustom#1{\colorpush{0 #1 0 0}\aftergroup\colorpop}
\def\YellowCustom#1{\colorpush{0 0 #1 0}\aftergroup\colorpop}

\def\bbb{\fam\bbfam}

\def\Af{{\frak A}} \def\Bf{{\frak B}} \def\Cf{{\frak C}} \def\Df{{\frak D}}
\def\Ef{{\frak E}} \def\Ff{{\frak F}} \def\Gf{{\frak G}} \def\Hf{{\frak H}}
\def\If{{\frak I}} \def\Jf{{\frak J}} \def\Kf{{\frak K}} \def\Lf{{\frak L}}
\def\Mf{{\frak M}} \def\Nf{{\frak N}} \def\Of{{\frak O}} \def\Pf{{\frak P}}
\def\Qf{{\frak Q}} \def\Rf{{\frak R}} \def\Sf{{\frak S}} \def\Tf{{\frak T}}
\def\Uf{{\frak U}} \def\Vf{{\frak V}} \def\Wf{{\frak W}} \def\Xf{{\frak X}}
\def\Yf{{\frak Y}} \def\Zf{{\frak Z}}

\def\Ac{{\cal A}} \def\Bc{{\cal B}} \def\Cc{{\cal C}} \def\Dc{{\cal D}}
\def\Ec{{\cal E}} \def\Fc{{\cal F}} \def\Gc{{\cal G}} \def\Hc{{\cal H}}
\def\Ic{{\cal I}} \def\Jc{{\cal J}} \def\Kc{{\cal K}} \def\Lc{{\cal L}}
\def\Mc{{\cal M}} \def\Nc{{\cal N}} \def\Oc{{\cal O}} \def\Pc{{\cal P}}
\def\Qc{{\cal Q}} \def\Rc{{\cal R}} \def\Sc{{\cal S}} \def\Tc{{\cal T}}
\def\Uc{{\cal U}} \def\Vc{{\cal V}} \def\Wc{{\cal W}} \def\Xc{{\cal X}}
\def\Yc{{\cal Y}} \def\Zc{{\cal Z}}

\def\As{{\scr A}} \def\Bs{{\scr B}} \def\Cs{{\scr C}} \def\Ds{{\scr D}}
\def\Es{{\scr E}} \def\Fs{{\scr F}} \def\Gs{{\scr G}} \def\Hs{{\scr H}}
\def\Is{{\scr I}} \def\Js{{\scr J}} \def\Ks{{\scr K}} \def\Ls{{\scr L}}
\def\Ms{{\scr M}} \def\Ns{{\scr N}} \def\Os{{\scr O}} \def\Ps{{\scr P}}
\def\Qs{{\scr Q}} \def\Rs{{\scr R}} \def\Ss{{\scr S}} \def\Ts{{\scr T}}
\def\Us{{\scr U}} \def\Vs{{\scr V}} \def\Ws{{\scr W}} \def\Xs{{\scr X}}
\def\Ys{{\scr Y}} \def\Zs{{\scr Z}}

\def\Ab{{\bbb A}} \def\Bb{{\bbb B}} \def\Cb{{\bbb C}} \def\Db{{\bbb D}}
\def\Eb{{\bbb E}} \def\Fb{{\bbb F}} \def\Gb{{\bbb G}} \def\Hb{{\bbb H}}
\def\Ib{{\bbb I}} \def\Jb{{\bbb J}} \def\Kb{{\bbb K}} \def\Lb{{\bbb L}}
\def\Mb{{\bbb M}} \def\Nb{{\bbb N}} \def\Ob{{\bbb O}} \def\Pb{{\bbb P}}
\def\Qb{{\bbb Q}} \def\Rb{{\bbb R}} \def\Sb{{\bbb S}} \def\Tb{{\bbb T}}
\def\Ub{{\bbb U}} \def\Vb{{\bbb V}} \def\Wb{{\bbb W}} \def\Xb{{\bbb X}}
\def\Yb{{\bbb Y}} \def\Zb{{\bbb Z}}

\def\Ae{{\eufm A}} \def\Be{{\eufm B}} \def\Ce{{\eufm C}} \def\De{{\eufm D}}
\def\Ee{{\eufm E}} \def\Fe{{\eufm F}} \def\Ge{{\eufm G}} \def\He{{\eufm H}}
\def\Ie{{\eufm I}} \def\Je{{\eufm J}} \def\Ke{{\eufm K}} \def\Le{{\eufm L}}
\def\Me{{\eufm M}} \def\Ne{{\eufm N}} \def\Oe{{\eufm O}} \def\Pe{{\eufm P}}
\def\Qe{{\eufm Q}} \def\Re{{\eufm R}} \def\Se{{\eufm S}} \def\Te{{\eufm T}}
\def\Ue{{\eufm U}} \def\Ve{{\eufm V}} \def\We{{\eufm W}} \def\Xe{{\eufm X}}
\def\Ye{{\eufm Y}} \def\Ze{{\eufm Z}}

\let\N=\Nb
\let\Z=\Zb
\let\Q=\Qb
\let\R=\Rb
\let\C=\Cb

\let\diff=\partial

\def\defeq{\mathbin{:=}}
\def\symdiff{\mathbin{\triangle}}
\def\sing{\mathrel{\bot}}
\def\uc{{\frak c}}

\let\emptyset=\varnothing

\def\Re{\mathop{\rm Re}}
\def\Im{\mathop{\rm Im}}
\def\sgn{\mathop{\rm sgn}}
\def\diam{\mathop{\rm diam}}
\def\supp{\mathop{\rm supp}}

\def\limsup{\mathop{\overline{\lim}}}
\def\liminf{\mathop{\vcenter{\hbox{$\underline{\lim}$}}}}
\let\lims=\limsup
\let\limi=\liminf

\def\cupdot{\mathbin{\ooalign{\hfil$\cup$\hfil\cr\hfil$\cdot$\hfil\cr}}}

\def\@bigcupdot#1#2#3#4{%
    \setbox0=\hbox{$#1\bigcup$}
    \setbox1=\hbox{\ooalign{\hfil$#1\bigcup$\hfil\cr\hfil\raise#3\hbox{$#2$}\hfil\cr}}
    \vcenter{\box1\kern#4\hbox{}}
}
\def\bigcupdot{\mathop{\mathchoice
    {\@bigcupdot{\displaystyle}{\scriptstyle\bullet}{1pt}{-8pt}}
    {\@bigcupdot{\textstyle}{\scriptscriptstyle\bullet}{1pt}{-10pt}}
    {\@bigcupdot{\scriptstyle}{\scriptscriptstyle\bullet}{0.7pt}{-12pt}}
    {\@bigcupdot{\scriptscriptstyle}{\cdot}{-1pt}{-12pt}}
}}

\def\unif@rm#1#2#3{\mathrel{\raise#2\hbox{$#1\rightarrow$}\mkern#3\lower#2\hbox{$#1\rightarrow$}}}
\def\uniform{%
    \mathchoice{\unif@rm\displaystyle{2.5pt}{-18mu}}
        {\unif@rm\textstyle{2.5pt}{-18mu}}
        {\unif@rm\scriptstyle{1.8pt}{-18mu}}
        {\unif@rm\scriptscriptstyle{1.2pt}{-17mu}}
}
\def\converges #1 to #2 with #3{%
    \ifx\uniform#3{#1}\uniform{#2}%
    \else{#1}\buildrel{#3}\over\to{#2}%
    \fi
}
\let\converge=\converges

\def\provedboxinit{\vbox{%
    \hrule\hbox{\vrule\kern 3pt\vbox{\kern 3pt\hbox{}\kern 3pt}%
    \kern 3pt\vrule}\hrule
}}

\def\lemmaprovedboxinit{\vrule height1.5ex width1.1ex}

\def\provedbox{%
    {\unskip\nobreak\hfil\penalty50
    \hfil\phantom{\provedboxinit}\nobreak\hfil\provedboxinit
    \parfillskip=0pt \finalhyphendemerits=0 \par}%
}

\def\proved{\ifmmode\eqno\hbox{\provedboxinit}\else\provedbox\fi}

\def\lemmaproved{%
    \ifmmode\eqno\hbox{\lemmaprovedboxinit}
    \else\hfill\lemmaprovedboxinit
    \fi
}

\def\Eqno{\global\advance\eqnocnt by 1 \eqno{(\the\sectioncnt.\the\subsectioncnt.\the\eqnocnt)}}
\def\Eqnolbl#1{\global\advance\eqnocnt by 1
\begingroup\edef\lblitem{(\the\sectioncnt.\the\subsectioncnt.\the\eqnocnt)}\label{#1}\endgroup\eqno{(\the\sectioncnt.\the\subsectioncnt.\the\eqnocnt)}}

\def\caption#1#2{\global\advance\captioncnt by 1
\begingroup\edef\lblitem{{\bf Figure \the\sectioncnt.\the\subsectioncnt.\the\captioncnt}}\label{#1}%
\centerline{{\bf Figure \the\sectioncnt.\the\subsectioncnt.\the\captioncnt:} #2}\endgroup}

\directlua{
function __MakeTheorem__(name)
    local string = require("string")
    local count = [[\string\newcount\string\]] .. name .. "cnt"
    local start_def = [[\string\def\string\]] .. string.lower(name) .. [[{
    \string\par\string\penalty-50\string\advance\string\]] .. name .. [[cnt by 1
    \string\begingroup
    \string\postdisplaypenalty=10000
    \string\vskip 1.5ex
    \string\hrule\space height 0.7pt\string\nobreak
    \string\vskip 1.5ex
    \string\noindent{\string\bf\space]] .. name ..
    [[\space\string\the\string\sectioncnt.\string\the\string\]] .. name .. [[cnt}\string\kern 1em
    \string\def\string\proof{%
        \string\par\string\hbox\space to\string\hsize{\string\xleaders\string\hbox\space
        to.8em{\string\hss-\string\hss}\string\hfill}
        \string\noindent{\string\it\space proof.}
    }
    \string\edef\string\lblitem{{\string\bf\space]] .. name ..
    [[\space\string\the\string\sectioncnt.\string\the\string\]] .. name .. [[cnt}}\string\relax
}]]

    local end_def = [[\string\def\string\end]] .. string.lower(name) .. [[{%
    \string\par\string\penalty10000
    \string\vskip1.5ex
    \string\hrule\space height 0.7pt
    \string\endgroup\string\par
    \string\vskip 1.5ex
}]]
    tex.print(count)
    tex.print(start_def)
    tex.print(end_def)
end

function __MakeTheoremWithName__(name)
    local string = require("string")
    local count = [[\string\newcount\string\]] .. name .. "cnt"
    local start_def = [[\string\def\string\name]] .. string.lower(name) .. [[\string#1{
    \string\par\string\penalty-50\string\advance\string\]] .. name .. [[cnt by 1
    \string\begingroup
    \string\postdisplaypenalty=10000
    \string\vskip 1.5ex
    \string\hrule\space height 0.7pt\string\nobreak
    \string\vskip 1.5ex
    \string\noindent{\string\bf\space]] .. name ..
    [[\space\string\the\string\sectioncnt.\string\the\string\]] .. name ..
    [[cnt}\space(\string#1)\string\kern 1em
    \string\def\string\proof{%
        \string\par\string\hbox\space to\string\hsize{\string\xleaders\string\hbox\space
        to.8em{\string\hss-\string\hss}\string\hfill}
        \string\noindent{\string\it\space proof.}
    }
    \string\edef\string\lblitem{{\string\bf\space]] .. name ..
    [[\space\string\the\string\sectioncnt.\string\the\string\]] .. name .. [[cnt}}\string\relax
}]]

    local end_def = [[\string\def\string\end]] .. string.lower(name) .. [[{%
    \string\par\string\penalty10000
    \string\vskip1.5ex
    \string\hrule\space height 0.7pt
    \string\endgroup\string\par
    \string\vskip 1.5ex
}]]
    tex.print(count)
    tex.print(start_def)
    tex.print(end_def)
end
}

\def\thmbox#1{%
    \par\penalty-50 \begingroup \postdisplaypenalty=10000 \vskip 1.5ex
    \hrule height0.7pt\nobreak \vskip 1.5ex
    \noindent{\bf#1.}\kern-5pt
    \def\proof{%
        \par\hbox to\hsize{\xleaders\hbox to.8em{\hss-\hss}\hfill}
        \noindent{\it proof.}
    }
    \edef\lblitem{{\bf#1}}\relax
}

\def\thmboxN#1#2{%
    \par\penalty-50 \begingroup \postdisplaypenalty=10000 \vskip 1.5ex
    \hrule height0.7pt\nobreak \vskip 1.5ex
    \noindent{\bf#1 (#2).}\kern-5pt
    \def\proof{%
        \par\hbox to\hsize{\xleaders\hbox to.8em{\hss-\hss}\hfill}
        \noindent{\it proof.}
    }
    \edef\lblitem{{\bf#1}}\relax
}
\def\endthmbox{\par\penalty10000 \vskip 1.5ex \hrule height0.7pt \endgroup\par \vskip 1.5ex }

\def\pf{%
    \par\hbox to\hsize{\xleaders\hbox to.8em{\hss-\hss}\hfill}
    \noindent{\it proof.}
}
\def\endpf{%
    \par\penalty10000\hbox to\hsize{\xleaders\hbox to.8em{\hss-\hss}\hfill}\vskip 1.5ex
}

\def\maketheorem#1{\directlua{__MakeTheorem__("#1")}}
\def\maketheoremwithname#1{\directlua{__MakeTheoremWithName__("#1")}}

\maketheorem{Definition}
\maketheorem{Theorem}
\maketheorem{Proposition}
\maketheorem{Corollary}
\maketheorem{Lemma}

\maketheoremwithname{Definition}
\maketheoremwithname{Theorem}
\maketheoremwithname{Proposition}
\maketheoremwithname{Corollary}
\maketheoremwithname{Lemma}

\let\defin=\definition
\let\enddefin=\enddefinition
\let\thm=\theorem
\let\endthm=\endtheorem
\let\prop=\proposition
\let\endprop=\endproposition
\let\coro=\corollary
\let\endcoro=\endcorollary

\let\namedefin=\namedefinition
\let\namethm=\nametheorem
\let\nameprop=\nameproposition
\let\namecoro=\namecorollary

\newread\aux
\immediate\openin\aux=\jobname.aux
\ifeof\aux \message{! No file \jobname.aux;}
\else \input \jobname.aux \immediate\closein\aux \fi
\newwrite\aux
\immediate\openout\aux=\jobname.aux

\def\strip#1>{}
\def\label#1{\immediate\write\aux%
{\string\expandafter\string\def\string\csname\space#1\string\endcsname%
{{\expandafter\strip\meaning\lblitem}}}}

\def\ref#1{\ifundefined{#1}{\bf [??]}\else\csname #1\endcsname\fi}

\catcode`@=12 % END OF coprime.tex
